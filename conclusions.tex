We have presented a novel way of adding coordination to a declarative language without
changing the nature of the language or introducing non-declarative constructs. We
took advantage of the fact that LM uses linear logic, which allows us to derive
coordination facts that can be deleted in order to perform actions in the underlying
runtime system. In the other hand, sensing facts allow the programmer to reason
about locality and scheduling of computation in a data-driven fashion.
We have presented several applications where a judicious use of coordination can lead to
better performance behavior.

We would like to explore this coordination paradigm in distributed systems,
where data locality is far more important than in shared memory systems. We also
think that LM may be useful as a glue distributed/parallel language since it can
easily model any computation pattern using graphs. Furthermore, we think that
other declarative paradigms would benefit from using a similar approach. For
instance, in functional programming, one may consider annotation functions that
pick the right granularity for each problem.
