We have presented a novel way of adding coordination to a declarative language without
changing the nature of the language or introducing non-declarative constructs. We
took advantage of the fact that CLM uses linear logic, which allows us to derive
coordination facts that can be deleted in order to perform actions in the underlying
runtime system. In the other hand, sensing facts allow the programmer to reason
about locality and scheduling of computation in a data-driven fashion.
Even though the programs are optimized, the proofs of correctness for CLM
programs remain simple since coordination does not change the language.

We would like to explore CLM's coordination principles in distributed systems,
where data locality is far more important than in shared memory systems.
Furthermore, we think that other declarative paradigms would benefit
from giving programmers the freedom to fine-tune their programs while keeping
the program declarative and easy to reason about.
