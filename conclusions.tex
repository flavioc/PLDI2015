We have presented a novel way of adding coordination to a declarative language without
changing the nature of the language or introducing non-declarative constructs. We
took advantage of the fact that CLM uses linear logic, which allows us to derive
coordination facts that can be deleted in order to perform actions in the underlying
runtime system. In the other hand, sensing facts allow the programmer to reason
about data locality and scheduling of computation in a data-driven fashion.
Even though the programs are optimized, programs remain declarative and proving
interesting properties of the programs remains simple because it is possible to
reason about the logical rules that drive coordination.

We would like to explore CLM's coordination principles in distributed systems,
where data locality is far more important than in shared memory systems.
Furthermore, we think that other declarative paradigms such as functional
programming would benefit from giving programmers the freedom to fine-tune their
programs.
