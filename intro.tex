Writing parallel programs in sequential languages is hard because manipulating
shared state using multiple threads may result in data race
conditions. Such issues are handled with low level constructs such as locks and
condition variables, requiring a fair amount of effort to get right.
Declarative programming has been hailed as a solution to this issue, since the
problem of implementing the details of parallelism is moved away from the programmer to the
compiler and runtime of the language. The programmer writes code
without having to deal with parallel programming constructs and the compiler
automatically parallelizes the program in order to take advantage of
multiple threads of execution.
The programming paradigm has been adopted with huge success in domain specific
languages such as SQL and MapReduce~\cite{Dean:2008:MSD:1327452.1327492}.
Although general declarative languages have yet to be as successful, the
future looks promising for this particular approach.

The problem with declarative programming is that it leaves little to no programmer control
over how execution is scheduled or how data is laid out, making it hard to achieve
necessary efficiency. This introduces some
performance issues because even if the runtime system is able to reasonably
parallelize the program using a general algorithm, there is a lack of specific
information about the program that a compiler does not easily understand. In
turn, such information would make execution better in terms of run time time,
memory usage and scalability.

In this paper, we extend Linear Meld (LM), a declarative language by
\cite{cruz-iclp14,cruz-ppdp14} with coordination facts that give programmer
control over scheduling and data placement. LM is a linear logic programming
language designed for programs that operate on graphs.  The use of linear
logic~\cite{girard-87} allows support for structured manipulation of mutable
state. In LM, computation is divided so that each node of the graph computes
independently and is allowed to communicate with other nodes.
Both computation and communication happen through derivation of logical rules.

Given that LM uses mutable state, the order in which nodes are executed may be
crucial in terms of performance. We extended the language with coordination
facts that can be used as a regular fact in the program logic to improve program
execution.  The first kind of coordination facts are called \emph{sensing facts}
and are used to sense information about the system the program is running on,
e.g., scheduling and node placement on threads. The second kind of coordination facts are
\emph{action facts} that when deleted in rules are used to apply a scheduling
operation during execution. Coordination facts allow the programmer to write
logical rules that depend on the current state of the program and then prioritize node
computation or place nodes in different threads.

To the best of our knowledge, this is the first time that a declarative language allows 
control over execution without a significant change in the language itself and
without using non-declarative operations. This is specially important since
programs can be proven to be correct even in the presence of coordination facts.
In order to prove our thesis, we first give a brief overview of the
base language, including a small example. Next, in
Section~\ref{sec:coordination} we introduce the available coordination facts,
followed by Section~\ref{sec:implementation}, where we describe the
implementation of the runtime required to implement the desired coordination
mechanisms. Afterwards, in Section~\ref{sec:applications} we present several applications where we add
programmer control to declarative programs in order to make them perform better.
