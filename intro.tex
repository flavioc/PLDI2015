Writing parallel programs in sequential languages is hard because manipulating
shared state using multiple threads may result in data race conditions. Such
issues are handled with low level constructs such as locks and condition
variables, requiring a fair amount of effort to get right.  Declarative
programming has been hailed as a solution to this issue, since the problem of
implementing the details of parallelism is moved from the programmer to the
compiler and runtime. The programmer writes code without having to deal with
parallel programming constructs and the compiler automatically parallelizes the
program in order to take advantage of multiple threads of execution.  The
programming paradigm has been adopted with huge success in domain specific
languages such as SQL and MapReduce~\cite{Dean:2008:MSD:1327452.1327492}.
Although general declarative languages have yet to be as successful, the future
looks promising for this particular approach.

The problem with declarative programming is that it leaves little to no
programmer control over how execution is scheduled or how data is laid out,
making it hard to achieve necessary efficiency. This introduces
performance issues because even if the runtime system is able to
reasonably parallelize the program using a general algorithm, there
is a lack of specific information about the program that a compiler
cannot easily deduce. Such information could make execution better in
terms of run time time, memory usage, and scalability.

In this paper, we introduce Coordinated Linear Meld (CLM), a declarative
language that extends the Linear Meld~(LM)
language~\cite{cruz-iclp14,cruz-ppdp14} with coordination facts that give
programmer control over scheduling and data placement. LM is a linear logic
programming language designed for programs that operate on graphs.  The use
of linear logic~\cite{girard-87} supports structured manipulation of mutable
state. In LM, computation is divided so that each node of the graph computes
independently and is allowed to \scare{communicate} with other nodes.  Both
computation and communication happen through the derivation of logical rules
(which make up the program).

The CLM language features coordination primitives that can be used in the same
way as any other primitive, in other words, they are specificied with the same
syntax and semantics as the rest of the programming language. These coordination
primitives can be used to improve program execution based on the state of the
program and the underlying machine. The first kind of coordination facts are
called \emph{sensing facts} and are used to sense information about the system
the program is running on, e.g., scheduling and node placement on threads. The
second kind of coordination facts are \emph{action facts} that when deleted in
rules are used to apply a scheduling operation during execution. Coordination
facts allow the programmer to write logical rules that depend on the current
state of the program and then prioritize node computation or place nodes in
different threads.

To the best of our knowledge, this is the first time that a declarative language
allows control over execution while staying declarative and without resorting to
meta-language constructs. This is crucial to our goal of being able to prove
programs correct since proofs can be constructed even in the presence of
coordination facts.  After briefly discussing related work, we present an
overview of the base language, with an example. In
Section~\ref{sec:coordination}, we introduce the current set of coordination
facts, followed by a description of the changes required to implement the
desired coordination mechanisms. In Section~\ref{sec:applications} we present
several applications and show coordination can improve programs without
destroying clarity or provability.
