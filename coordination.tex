The SSSP program is concise and declarative but its performance depends on the order in which nodes are
executed. If nodes with greater distances are prioritized over other nodes, the
program will generate more \mytt{relax} facts since it will take longer to
reach the shortest distances. From Fig.~\ref{fig:shortest_path_program}, it is
clear that the best scheduling is the following: \mytt{@1},
\mytt{@3}, \mytt{@2} and then \mytt{@4}, where only 4 \mytt{relax}
facts are generated. If we had decided to process nodes in order
\mytt{@1}, \mytt{@2}, \mytt{@4}, \mytt{@3}, \mytt{@4},
\mytt{@2}, then 6 \mytt{relax} facts would have been generated.
The optimal solution for SSSP is to schedule the node with the
shortest distance, which is essentially the Dijkstra shortest path
algorithm~\cite{Dijkstra}. Note how it is possible to change the nature of
the algorithm by simply changing the order of node computation, but still
retain the declarative nature of the program.

\emph{Coordination facts} allow the programmer
to change how the run time schedules nodes and how it partitions the nodes among
threads of execution. Coordination facts can be used in either the body of the
rule, the head of the rule or both.
This allows scheduling and partition decisions to be made based on the state of
the program and on the state of the underlying machine.
In this fashion, we keep the language declarative because we reason logically
about the state of execution, without the need to introduce extra-logical
operators into the language that would introduce significant issues when proving
properties about programs.

There are two kinds of coordination facts: \emph{sensing} and \emph{action}
facts. Sensing facts are used to sense information about the
underlying runtime system, including node placement and node scheduling.
Action facts are used to apply a coordination operations on the runtime system.

\subsection{Scheduling Facts}\label{sec:fifo}

In order to allow different scheduling strategies, we introduce the concept of
\emph{node priority} by assigning a priority value to every node in the program
and by introducing coordination facts that manipulate such priority values.
By default, nodes have no priority and can be picked in any order. In
our implementation, we use a FIFO approach because
older nodes tend to a higher number of unexamined facts, from which to derive
subsequent new facts.

We have two kinds of priorities: a \emph{temporary priority} and a \emph{default
priority}. A temporary priority momentarily changes the default priority $D$ of a
node, so that once the node is done, the priority will default back to $D$.
Initially, all nodes have a default priority of $0$.

The following list presents the action facts available to manipulate the scheduling
decisions of the system:

\newcommand{\code}[1]{\mytt{\small{#1}}}

\begin{tightitemize}
   \item \code{set-priority(node A, float F)}: This sets the
   temporary priority of \mytt{A} to \mytt{F}. If \mytt{A} has a priority
   \mytt{F'}, we only change the priority if \mytt{F} is higher. The programmer
   can decide if priorities are to be ordered in ascending or descending order.
   \item \code{add-priority(node A, float F)}: Increases,
   temporarily, the priority of node \mytt{A} by \mytt{F}.
   \item \code{remove-priority(node A)}: Removes the temporary priority from node
   \mytt{A}.
   \item \code{schedule-next(node A)}: Changes the temporary priority of node
   \mytt{A} to be 1 unit higher than the current highest priority.
   \item \code{set-default-priority(node, float)}: Sets the default
   priority of the node.
   \item \code{stop-program(node A)}: Immediately stops the
   execution of the whole program.
\end{tightitemize}

CLM provides the sensing fact \code{priority(node A, node B, float P)} in order to sense the
priority \mytt{P} of node \mytt{B} (at node \mytt{A}).
Sensing facts are only used in the body of the rules in order
to fetch information from the runtime system.
Note that when sensing facts are used to prove new facts, they are re-derived automatically.
~\footnote{The observant reader will notice that all the coordination facts are
linear and when created in the body of a rule will be consumed. The system
creates the necessary code to re-derive them without programmer interaction.
Likewise, \mytt{set-priority} and \mytt{set-default-priority} update the
value of \mytt{priority} facts by consuming and re-deriving them but this is
done automatically by the runtime system.}

\subsection{Partitioning facts}

We provide several coordination facts for dealing with node partitioning among
the running threads. In terms of action facts, we have the following:

\begin{tightitemize}
   \item \code{set-cpu(node A, int T)}: Moves node \mytt{A} to thread
   \mytt{T}.
   \item \code{set-affinity(node A, node B)}: Places node \mytt{B} in
   the thread of node \mytt{A}.
   \item \code{set-moving(node A)}: Allows node \mytt{A} to move freely
   between threads.
   \item \code{set-static(node A)}: Forces node \mytt{A} to stay in the
   same thread indefinitely.
\end{tightitemize}

For partitioning facts, we have the following set of sensing facts:

\begin{tightitemize}
   \item \code{cpu-id(node A, node B, int T)}: Store at node \mytt{A} which
   thread \mytt{T}, \mytt{B} is actually running on.
   \item \code{moving(node A, node B)}: Fact available at node \mytt{A} if \mytt{B} is allowed
   to move between threads.
   \item \code{static(node A, node B)}: Fact available at node \mytt{A} if \mytt{B} is not
   allowed to move between threads.
\end{tightitemize}

\iffalse
\subsubsection{Global Directives}

We also provide a few global coordination statements:

\begin{tightdescription}
   \item[\mytt{priority @order ORDER.}] \mytt{ORDER} can be either \mytt{asc} or \mytt{desc}. This defines if node's are to be selected by the smallest or the greatest priority, respectively.
   \item[\mytt{priority @initial P.}] The \mytt{initial} statement informs the runtime system that all nodes must start with priority $P$. Alternatively, the programmer can define an \mytt{set-priority(A, P)} axiom.
   \item[\mytt{priority @static.}] The \mytt{static} priority tells the runtime system that the partition of nodes among workers is to be used until the end of program. 
\end{tightdescription}

\fi
