\subsection{Absolute runtime}

The measures of absolute run time performance for the VM were done
~\cite{cruz-ppdp14}.
To put our VM in perspective, we first compare it in terms of absolute

execution time with other competing systems using a single thread.

In Table~\ref{tbl:comp_nqueens}, we compare LM's version of the
classic N-Queens puzzle against 3 other versions: a straightforward
sequential program implemented in C using backtracking; a sequential
Python implementation~\cite{vanRossum:1995:PRM}; and a Prolog
implementation executed in YAP
Prolog~\cite{DBLP:journals/corr/abs-1102-3896}, an efficient
implementation of Prolog. Numbers less than 1 mean that LM is faster
and larger than 1 mean that LM is slower. We can observe that LM
easily beats Python, but is 5 to 10 times slower than YAP Prolog and
around 15 times slower than C. Note however that, as we will see next,
if we use at least 16 threads in LM, we can beat the sequential
implementation written in C.

\begin{table}[ht]
\centering
{\begin{tabular}{c|c|c|c}
\textbf{Problem} & \multicolumn{3}{c}{\textbf{System}} \\
\textbf{Size} & \textbf{C} & \textbf{Python} & \textbf{YAP Prolog} \\
\hline\hline
\textbf{10x10} & 16.92 & 0.62 & 5.42 \\
\textbf{11x11} & 21.59 & 0.64 & 6.47 \\
\textbf{12x12} & 10.32 & 0.73 & 7.61 \\
\textbf{13x13} & 14.35 & 0.88 & 10.38 \\
\end{tabular}}
\caption{Comparing the absolute execution times (LM/System) for the
  N-Queens program}
\label{tbl:comp_nqueens}
\end{table}

In Table~\ref{tbl:comp_bp}, we compare LM's Belief Propagation (BP)
program, a machine learning algorithm to denoise images, against a
sequential C, Python and GraphLab~\cite{GraphLab2010} version of the
algorithm. GraphLab is a parallel C++ library used to solve
graph-based problems in machine learning. C and GraphLab perform about
the same since they are both compiled to machine code, although
the GraphLab version is highly optimized to run on multicore architectures.
Python runs very slowly since it is a
dynamic programming language and BP has many mathematical
computations. We should note, however, that LM's version uses some
external functions written in C++ in order to improve execution time,
therefore the comparison is not totally fair.

\begin{table}[ht]
\centering
{\begin{tabular}{c|c|c|c}
\textbf{Problem} & \multicolumn{3}{c}{\textbf{System}} \\
\textbf{Size} & \textbf{C} & \textbf{Python} & \textbf{GraphLab} \\
\hline\hline
\textbf{10}  & 1.00 & 0.03 & 1.00 \\
\textbf{50}  & 1.77 & 0.04 & 1.73 \\
\textbf{200} & 1.99 & 0.05 & 1.79 \\
\textbf{400} & 2.00 & 0.04 & 1.80 \\
\end{tabular}}
\caption{Comparing the absolute execution times (LM/System) for the
  Belief Propagation program}
\label{tbl:comp_bp}
\end{table}

We also compared a LM's version of the PageRank program against a
similar GraphLab version and LM showed to be around 4 to 6 times
slower. Our worse results were obtained for the all-pairs shortest
distance algorithm where a LM's version of the problem was around 50
times slower than a C sequential implementation of the Dijkstra
algorithm, but almost twice as fast when compared to the same
implementation in Python.

