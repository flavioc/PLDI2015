We start by adding coordination to the SSSP program described before.
The coordinated version of the
SSSP~(Fig.~\ref{code:shortest_path_program_coord}) uses the coordination fact
\mytt{set-priority} (line 14) and a global program directive to order
priorities in ascending order (line 5).

When run with one thread, the algorithm behaves like
Dijkstra's shortest path algorithm~\cite{Dijkstra}. When using multiple
threads, each thread will pick the shortest distance from their subset of nodes.
While this does not yield the optimal program with relation to 1 thread, it
allows for parallel execution and locally avoids unnecessary work. The result
scales well and it is close to Dijkstra's algorithm.

\begin{topfig}
\scriptsize\begin{Verbatim}[numbers=left,xleftmargin=7mm,commandchars=\\\{\}]
type route edge(node, node, int).
type linear shortest(node, int, list int).
type linear relax(node, int, list int).

\underline{priority @order asc}.

shortest(A, +00, []).
relax(@1, 0, [@1]).

shortest(A, D1, P1), D1 > D2, relax(A, D2, P2)
   -o shortest(A, D2, P2),
      \{B, W | !edge(A, B, W) |
         relax(B, D2 + W, P2 ++ [B]),
         \underline{set-priority(B, float(D2 + W))}\}.

shortest(A, D1, P1), D1 <= D2, relax(A, D2, P2)
   -o shortest(A, D1, P1).
\end{Verbatim}
  \scap{code:shortest_path_program_coord}{Shortest Path Program.}
\end{topfig}
\normalsize

The most interesting property of the SSSP program presented in
Fig.~\ref{code:shortest_path_program_coord} is that it remains provably correct,
although it applies rules using smarter ordering and the code remains
declarative. Since the proof of correctness considers that, eventually, the shortest
path is computed at all nodes of the graph, the use of \mytt{set-priority}
does not change this at all.

Fig.~\ref{results:sssp_uspowergrid} shows experimental results when
the program computes the SSSP for 20\% of the nodes of the graph.  The
coordinated version scales well and when compared to the regular version
using 16 threads, it is 1.3 times faster. The coordinated version
also produces 40\% fewer facts with any number of threads.
Note that there are some situations where unnecessary facts are
propagated because although the shortest distance is selected locally,
sub-optimal distances may be propagated because many SSSP distances
are computed at the same time.

\begin{topfig}
\vspace*{-5ex}
   \begin{center}
      \includegraphics[width=\figsize]{results/shortest-uspowergrid.png}
   \end{center}
   \scap{results:sssp_uspowergrid}{Scalability results for SSSP run on the 
   US power grid network. The speedup values are computed using the
   execution time for 1 thread.}
\vspace*{-3ex plus 0pt minus 1ex}
\end{topfig}
